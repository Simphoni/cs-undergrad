\documentclass[./main.tex]{subfiles}
\begin{document}

\chapter{线性代数}

\section{线性映射和矩阵}

\subsection{线性映射}

\tinysec{定义}\concept{线性运算}指向量的加法与数乘。\\

\tinysec{定义}\concept{向量空间}为带有线性运算的集合 $\mathbb{R}^m$ 被称为向量空间。\\

\tinysec{定义}映射 $f:\linsp{n}\rightarrow\linsp{m}$ 若满足
\begin{enumerate*}
    \item 任意 $\bm{x},\bm{x}'\in\linsp{n}$,都有 $f(\bm{x}+\bm{x'})=f(\bm{x})+f(\bm{x'})$
    \item 任意 $\bm{x}\in\linsp{n},k\in\linsp{}$, 都有 $f(k\bm{x})=kf(\bm{x})$
\end{enumerate*}
则 $f$ 为 从 $\linsp{n}$ 到 $\linsp{m}$ 的 \concept{线性映射}。\\

\tinysec{定义}从 $\linsp{n}$ 到 $\linsp{n}$ 的线性映射称为\concept{线性变换}.\\

\tinysec{定义}若线性映射 $f$ 有 $f(\bm{e_i})=\bm{a_i},\bm{e_i}\in\linsp{n}, \bm{a_i}\in\linsp{m}$,则矩阵 $A=[\bm{a_1},\dots,\bm{a_n}]$ 即为标准坐标向量下的 \concept{线性映射的表示矩阵},且满足 $A\bm{e_i}=\bm{a_i}$。\\

\tinysec{定理}(线性映射的线性运算)若矩阵 $\bm{A},\bm{B}$ 表示 $\linsp{n}\rightarrow\linsp{m}$ 的线性映射,则 $\bm{A}+\bm{B}$ 与 $k\bm{A}$ 也是同样范畴上的线性映射。\\

\tinysec{定理}$\bm{AB}$ 表示线性映射 $\bm{A},\bm{B}$ 的复合 $\bm{A}\circ\bm{B}$。需要注意,$\bm{AB}=0$ 也不能推出 $\bm{A}=0$ 或 $\bm{B}=0$。\\

\tinysec{理解}矩阵乘法 $A\bm{x}$ 视为这样一种运算:按照 $\bm{x}$ 指定的系数,将 $\bm{A}$ 的列进行线性组合。\\

\tinysec{定义}\concept{反对称矩阵}:$A=-A^{\mathrm{T}}$,反对称矩阵对角线必定为0。\\

\tinysec{定义}若一个上三角矩阵对角线上全为0,则称为\concept{严格上三角矩阵}。\\

\tinysec{定义}将阶梯形矩阵,从下向上消元,并单位化\concept{主元},得到的矩阵每个非零行上,主变量为 1 而其他列的元素均为 0,称这个矩阵为 \concept{行简化阶梯形矩阵}。\\

\tinysec{定理}对于 $\bm{v},\bm{w}\in\linsp{m}$,$\bm{v}^{\mathrm{T}}\bm{w} = trace(\bm{w}\bm{v}^{\mathrm{T}})$。

\subsection{线性方程式组}

\tinysec{定理}对于方程组 $A\bm{x}=\bm{b}$,将 $[A\ \bm{b}]$ 简化为阶梯型后,若

\begin{enumerate*}
    \item $[A\ \bm{b}]$ 阶梯数比 $A$ 多\textbf{1},则方程组无解($0\neq 1$)
    \item $[A\ \bm{b}]$ 阶梯数与 $A$ 相等,则方程组有解
    \begin{enumerate*}
        \item 若阶梯数等于未知数个数,则有唯一解
        \item 若阶梯数小于未知数个数,则有无穷多组解
    \end{enumerate*}
\end{enumerate*}

\tinysec{定义}$A\bm{x}=\bm{b}$ 称为\concept{齐次线性方程组},$\vec{0}$ 为其\concept{平凡解},除此之外的解称为\concept{非平凡解}。

\subsection{可逆矩阵}

\tinysec{定义}设 $A$ 为 $n$ 阶方阵,若存在 $n$ 阶方阵 $B$,使得 $AB=BA=I_n$,则 $A$ 为 \concept{可逆矩阵} 或 \concept{非奇异矩阵},$B$ 为 $A$ 的逆。\\

\tinysec{定理}以下命题等价
\begin{enumerate*}
    \item $A$ 可逆
    \item 任意 $\bm{b}\in\linsp{n}$,$A\bm{x}=\bm{b}$ 的解唯一
    \item 其次方程组 $A\bm{x}=0$ 仅有零解
    \item $A$ 对应的阶梯形矩阵有 $n$ 个主元
    \item $A$ 对应的行简化阶梯形矩阵是 $I_n$
    \item $A$ 能表示为有限个初等矩阵的乘积(即消元至行简化阶梯形矩阵的逆过程)
\end{enumerate*}

\tinysec{定义}若矩阵 $A=[a_{ij}]_{n\times n}$ 对于 $i=1,2,\dots,n$ 都有 $|a_{ii}|>\sum_{j\neq i}|a_{ij}|$,则称其为 \concept{(行)对角占优矩阵}。\\

\tinysec{定理}对角占优矩阵必然可逆。\\

\tinysec{定义}若矩阵 $A$ 通过若干初等\textbf{行变换}可以变为矩阵 $B$,则称 $A,B$ \concept{左相抵}。即存在可逆矩阵 $P$,使得 $PA=B,A=P^{-1}B$。所有和 $A$ 相抵的矩阵中,最简单的是其行简化阶梯形,它被称为 $A$ 的\concept{左相抵标准形}。\\

\tinysec{定理}左相抵构成等价关系。\\

\tinysec{定理}\textbf{(Sherman-Morrison)}设 $A$ 为 $n$ 阶可逆方阵,$\bm{u},\bm{v}$ 为 $n$ 阶向量,则 $A+\bm{uv}^{\mathrm{T}}$ 可逆 $\iff 1+\bm{v}^{\mathrm{T}}A^{-1}\bm{u}\neq 0$,且此时
\begin{equation}
    (A+\bm{uv}^{\mathrm{T}})^{-1}=A^{-1}-\frac{A^{-1}\bm{uv}^{\mathrm{T}}A^{-1}}{1+\bm{v}^{\mathrm{T}}A^{-1}\bm{u}}
\end{equation}
若将 $\bm{u},\bm{v}$ 改为 $n\times k$ 的矩阵,则类似地有
\begin{equation}
    (A+\bm{uv}^{\mathrm{T}})^{-1}=A^{-1}-A^{-1}\bm{u}(I_k+\bm{v}^{\mathrm{T}}A^{-1}\bm{u})^{-1}\bm{v}^{\mathrm{T}}A^{-1}
\end{equation}

\subsection{LU分解}
\tinysec{定理}若 $n$ 阶方阵 $A$ 仅通过倍加矩阵做行变化即可化为阶梯形,则存在\concept{单位下三角矩阵}(主对角线均为1)$L$ 与上三角矩阵 $U$,使得 $A=LU$,此即 \concept{LU分解}。\\

\tinysec{定义}方阵 $A$ 左上角的 $k\times k$ 块为第 $k$ 个\concept{顺序主子阵}。\\

\tinysec{定理}可逆矩阵 $A$ 存在 LU 分解,当且仅当 $A$ 的所有顺序主子阵均可逆,此时 LU 分解唯一。(满足在消元过程中不需要行的调换)\\

\tinysec{定理}若可逆矩阵 $A$ 存在 LU 分解,则存在对角线均不为 0 的对角矩阵 $D$、单位下三角矩阵 $L$、单位上三角矩阵 $U$,满足 $A=LDU$,且该分解\textbf{唯一}。此即 \concept{LDU分解}。\\

\tinysec{定理}若可逆对称阵 $A$ 有 LDU 分解,则 $L=U^{\mathrm{T}}$。\\

\tinysec{定理}可逆矩阵 $A$ 存在分解 $A=PLU$,$P$ 为置换矩阵,显然该分解不唯一。\\

\tinysec{技巧}将 $A$ 分解成 对称阵 $X=\frac{1}{2}(A+A^{\mathrm{T}})$ 与 反对称阵 $Y=\frac{1}{2}(A-A^{\mathrm{T}})$。

\section{子空间和维数}

\subsection{基本概念}

\tinysec{定义}映射 $A:\bm{x}\mapsto A\bm{x}\in\linsp{m\times n}$ 的\concept{像集}
\begin{equation}
    \mathcal{R}(A)=\{A\bm{x}|\bm{x}\in\linsp{n}\}\subseteq\linsp{m}
\end{equation}
显然 $\mathcal{R}(A)$ 必是子空间,即 $A$ 的\concept{列(向量)空间}。\\

\tinysec{命题} $\mathcal{R}(A)=\linsp{m}\iff A$ 为满射。\\

\tinysec{定义}映射 $A:\bm{x}\mapsto A\bm{x}\in\linsp{m\times n}$ 的\concept{原像}
\begin{equation}
    \mathcal{N}(A)=\{\bm{x}\in\linsp{n}|A\bm{x}=\bm{0}\}\in\linsp{n}
\end{equation}
显然 $\mathcal{N}(A)$ 必是子空间,即 $A$ 的\concept{零空间(也称解空间)}。\\

\tinysec{命题} $\mathcal{N}(A)=\{\bm{0}\}\iff A$ 是单射 $\iff \forall\bm{x}_1\neq\bm{x}_2,\ A(\bm{x}_1-\bm{x}_2)\neq 0$。\\

\tinysec{定义}将 $\bm{a}_1,\dots,\bm{a}_n$ 的全体线性组合为 $\linsp{n}$ 的子空间,记为
\begin{equation}
    \mathrm{span}(\bm{a}_1,\dots,\bm{a}_n):=\{k_1\bm{a}_1,\dots,k_n\bm{a}_n|k_1,\dots,k_n\in\mathbb{R}\}
\end{equation}

\tinysec{定理}$A=\big[\bm{a}_1\ \cdots\ \bm{a}_n\big]\in\linsp{n\times n}$ 可逆 $\iff \bm{a}_1,\dots,\bm{a}_n$ 是 $\linsp{n}$ 的一组基。

\subsection{基和维数}
\tinysec{定义}向量组 $S$ 的任一\concept{极大线性无关组}中向量的个数称为 $S$ 的\concept{秩(rank)};子空间 $\mathcal{M}$ 的一组基的向量数目称为\concept{维数},记为 $\mathrm{dim}\ \mathcal{M}$。\\

\tinysec{定理}\concept{(基存在定理)}给定 $\linsp{m}$ 的子空间 $\mathcal{M}\neq\{0\}$,则 $\mathcal{M}$ 存在一组基,且基向量个数不超过 $m$。\\

\tinysec{定理}\concept{(基扩充定理)}若 $\mathcal{M}\subseteq\mathcal{N}$,则 $\mathcal{M}$ 的任意一组基都能够\concept{扩充}成 $\mathcal{N}$ 的一组基。\\

\tinysec{定理}对于方阵 $A\in\linsp{n\times n}$,$A$ 可逆 $\iff$ $A$ 是单射 $\iff$ $A$ 是满射。

\subsection{矩阵的秩}
\tinysec{定义}执行 Gauss 消去法将矩阵化为阶梯型后,主元所在的列对应主变量的系数,称为\concept{主列};其他列对应自由变量的系数,称为\concept{自由列}。\\

\tinysec{定理}若矩阵 $A=\big[\bm{a}_1\ \dots\ \bm{a}_n\big]$ 与 $B=\big[\bm{b}_1\ \dots\ \bm{b}_n\big]$ 左相抵,则

\begin{enumerate*}
    \item 部分组 $\bm{a}_{i_1},\dots,\bm{a}_{i_r}$ 线性无关,当且仅当 $\bm{b}_{i_1},\dots,\bm{b}_{i_r}$ 线性无关。
    \item $\bm{a}_j=k_1\bm{a}_{i_1}+\cdots+k_r\bm{a}_{i_r}\iff\bm{b}_j=k_1\bm{b}_{i_1}+\cdots+k_r\bm{b}_{i_r}$
\end{enumerate*}

\tinysec{二级结论}
\begin{itemize*}
    \item $\mathrm{rank}(A)+\mathrm{rank}(B)=\mathrm{rank}\left(\begin{bmatrix}A &O\\ O &B\end{bmatrix}\right)\le\mathrm{rank}\left(\begin{bmatrix}A &X\\ O &B\end{bmatrix}\right)$。因而当 $A,B$ 可逆时,上述两个分块矩阵都可逆
    \item $\mathrm{rank}(A+B)\le\mathrm{rank}(A)+\mathrm{rank}(B)$
    \item $\mathrm{rank}(AB)\le\min\{\mathrm{rank}(A),\mathrm{rank}(B)\}$
    \item 反对称矩阵的秩必定是偶数
    \item $\mathcal{R}(AB)\subseteq\mathcal{R}(A)$
\end{itemize*}

\tinysec{定义}$\mathcal{R}(A^{\mathrm{T}})$ 为 $A$ 的\concept{行(向量)空间};$\mathrm{rank}(A)=\mathrm{rank}(A^{\mathrm{T}})$;$\mathrm{rank}(A)=n$ 定义为\concept{列满秩},同理有\concept{行满秩}。\\

\tinysec{定理}对于 $A\in\linsp{m\times n}$,
\begin{itemize*}
    \item $A$ 是满射 $\iff$ $A$ 行满秩
    \item $A$ 是单射 $\iff$ $A$ 列满秩
\end{itemize*}

\tinysec{定义} 通过行列初等变换,能将矩阵 $A$ 变换为\concept{相抵标准型} $\begin{bmatrix} I_r & O\\ O &O\end{bmatrix}$。

\subsection{线性方程组的解}
\tinysec{方法}(求零空间的一组基)矩阵 $A\in\linsp{m\times n}$,求其行简化阶梯形,考虑依次将各个自由元取为 1,同时维持其他自由元为 0(即只让其中一个自由列在线性组合中系数非零),由于主列都是单位向量,从而容易根据自由列的系数计算出零解。这样求出的解(由于自由元的取值特点)构成零空间的一组基,继而有 $\mathrm{dim}\ \mathcal{N}(A)=n-\mathrm{rank}(A)$。

\section{内积和正交性}

\subsection{基本概念}

\tinysec{定义}$\bm{b}$ 向 $\bm{a}$ 的\concept{垂直投影} 为 $\hat{x}=\frac{\bm{a}^{\mathrm{T}}\bm{b}}{\bm{a}^{\mathrm{T}}\bm{a}}$;向量 $\frac{\bm{a}^{\mathrm{T}}\bm{b}}{\bm{a}^{\mathrm{T}}\bm{a}}\bm{a}$ 称为向量 $\bm{b}$ 向直线 $\mathrm{span}(\bm{a})$ 的投影。\\

\tinysec{定理}\concept{(Cauchy-Schwarz 不等式)} $|\bm{a}^{\mathrm{T}}\bm{b}|\le \|a\|\|b\|$。\\

\tinysec{定义}设 $\mathcal{M}$ 是 $\linsp{n}$ 的子空间,若它的一组基是\concept{正交向量组},则称之为 $\mathcal{M}$ 的一组\concept{正交基};若是\concept{正交单位向量组},则称为 $\mathcal{M}$ 的\concept{标准正交基}。\\

\tinysec{定义}\concept{(Gram-Schmidt 正交化)}从 $\mathcal{M}$ 的任意一组基 $\bm{a}_1,\dots,\bm{a}_r$ 出发,执行如下操作
\begin{equation}
    \bm{\widetilde{q}}_k=\bm{a}_k-\sum_{j=1}^{k-1}\frac{\bm{\widetilde{q}}_j^{\mathrm{T}}\bm{a}_k}{\bm{\widetilde{q}}_j^{\mathrm{T}}\bm{\widetilde{q}}_j}\bm{\widetilde{q}}_j \ \ \ (k=1,2,\dots,r)
\end{equation}
即得到一组正交基,最后再通过单位化,得到一组标准正交基。

\subsection{正交矩阵和 QR 分解}

\tinysec{定义}若方阵 $Q$ 满足 $Q^{\mathrm{T}}Q=I_n$,则称 $Q$ 为 $n$ 阶\concept{正交矩阵};$Q$ 的行、列向量各自构成 $\linsp{n}$ 的一组标准正交基;多个正交矩阵的积也为正交矩阵。\\

\tinysec{定理}方阵 $Q$ 的以下叙述等价
\begin{enumerate*}
    \item $Q$ 是正交矩阵
    \item $Q$ 为\concept{保距变换},即 $\|Q\bm{x}\|=\|\bm{x}\|$
    \item $Q$ 为\concept{保内积变换},即 $Q\bm{x}\cdot Q\bm{y}=\bm{x}\cdot\bm{y}$
\end{enumerate*}
保距变换一定也是\concept{保角变换}。\\

\tinysec{定义}\concept{(Givens 变换)}在 $\bm{e}_i-\bm{e}_j$ 平面上转角 $\theta$ 的旋转变换的矩阵为

\begin{equation}
    \begin{bmatrix}
        \begin{smallmatrix}
            \ddots &&&&&&&&\\
            &1&&&&&&&\\
            &&\cos\theta&&&&-\sin\theta&&\\
            &&&1&&&&&\\
            &&&&\ddots &&&&\\
            &&&&&1&&&\\
            &&\sin\theta&&&&\cos\theta&&\\
            &&&&&&&1&\\
            &&&&&&&&\ddots
        \end{smallmatrix}
    \end{bmatrix}
\end{equation}

\tinysec{定义}\concept{(Householder 变换)}关于 $\linsp{n}$ 中的单位法向量所确定的超平面 $\mathcal{N}(\bm{v}^{\mathrm{T}})$ 进行反射变换的矩阵为
\begin{equation}
    H_{\bm{v}}=I_n-2\bm{v}\bm{v}^{\mathrm{T}}
\end{equation}
不难注意到 $\bm{v}\bm{v}^{\mathrm{T}}\bm{w}$ 是 $\bm{w}$ 向 $\mathrm{span}(\bm{v})$ 的投影,因此 $H_{\bm{v}}$ 的效果是将 $\bm{w}$ 中与 $\bm{v}$ 共线的成分反向。\\

\tinysec{定义}\concept{(QR 分解)}设 $A=\begin{bmatrix}\bm{a}_1&\dots&\bm{a}_n\end{bmatrix}$ 为 $n$ 可逆矩阵。在对 $\{\bm{a}_1,\dots,\bm{a}_n\}$ 进行 Gram-Schmidt 正交化过程的中,本质上是进行多次的列初等变换,故可以借此对 $A$ 进行表达

\begin{equation}
    \bm{a}_k=\bm{\widetilde{q}}_k+\sum_{j=1}^{k-1}\frac{\bm{\widetilde{q}}_j^{\mathrm{T}}\bm{a}_k}{\bm{\widetilde{q}}_j^{\mathrm{T}}\bm{\widetilde{q}}_j}\bm{\widetilde{q}}_j \ \ \ (k=1,2,\dots,r)
\end{equation}
\begin{equation}
    A=\widetilde{Q}\widetilde{R}=\begin{bmatrix}\widetilde{\bm{q}}_1&\dots&\widetilde{\bm{q}}_n\end{bmatrix}
    \begin{bmatrix}
        1 & \frac{\widetilde{\bm{q}}_1^{\mathrm{T}}\bm{a}_2}{\widetilde{\bm{q}}_1^{\mathrm{T}}\widetilde{\bm{q}}_1} & \cdots & \frac{\widetilde{\bm{q}}_1^{\mathrm{T}}\bm{a}_n}{\widetilde{\bm{q}}_1^{\mathrm{T}}\widetilde{\bm{q}}_1}             \\
          & 1                                                                                                       & \ddots & \vdots                                                                                                              \\
          &                                                                                                         & \ddots & \frac{\widetilde{\bm{q}}_{n-1}^{\mathrm{T}}\bm{a}_n}{\widetilde{\bm{q}}_{n-1}^{\mathrm{T}}\widetilde{\bm{q}}_{n-1}} \\
          &                                                                                                         &        & 1
    \end{bmatrix}
\end{equation}

可以进一步将 $Q$ 单位化,得到

\begin{equation}
    A=Q\ \mathrm{diag}(\|\widetilde{\bm{q}}_i\|)\widetilde{R}=QR
\end{equation}

\tinysec{定理}设 $A$ 为 $n$ 阶可逆矩阵,则存在唯一的分解 $A=QR$,其中 $Q$ 为正交矩阵,$R$ 为对角元均为正数的上三角矩阵。\\

\tinysec{定义}若矩阵 $Q$ 满足 $Q^{\mathrm{T}}Q=I_n$,则称之为 \concept{列正交矩阵}。\\

\tinysec{定理}对于 $m\times n$ 的矩阵 $A$,其中 $m\ge n$,则
\begin{enumerate*}
    \item \concept{(简化 QR 分解)}存在 $m\times n$ 列正交矩阵 $Q_1$ 和具有非负对角元的 $n$ 阶上三角矩阵 $R_1$,使得 $A=Q_1R_1$
    \item \concept{(QR 分解)}进一步的,存在 $m$ 阶正交矩阵 $Q$ 和 $m\times n$ 矩阵 $R$,使得 $A=QR=\begin{bmatrix}Q_1&Q_2\end{bmatrix}\begin{bmatrix}R_1\\O\end{bmatrix}$
\end{enumerate*}

\subsection{正交投影}
\tinysec{定义}\concept{(子空间正交)}若子空间 $\mathcal{M}$ 的任意向量 与 $\mathcal{N}$ 中的任意向量都正交,则称为 $\mathcal{M}$ 与 $\mathcal{N}$ 正交,记为 $\mathcal{M}\perp\mathcal{N}$。\\\indent 若 $\mathcal{M},\mathcal{N}\in\linsp{n},\mathcal{M}\perp\mathcal{N},\mathcal{M}\cup\mathcal{N}=\linsp{n}$,则称 $\mathcal{N}$ 为 $\mathcal{M}$ 的\concept{正交补},记为 $\mathcal{N}=\mathcal{M}^{\perp}$。\\

\tinysec{定理}对于 $A\in\linsp{m\times n}$,有
\begin{enumerate*}
    \item $\mathcal{R}(A^{\mathrm{T}})^{\perp}=\mathcal{N}(A),\mathcal{R}(A)^{\perp}=\mathcal{N}(A^{\mathrm{T}})$ (矩阵导出的四个子空间的关系)
    \item $\mathcal{R}(A^{\mathrm{T}}A)=\mathcal{R}(A^{\mathrm{T}}),\mathcal{N}(A^{\mathrm{T}}A)=\mathcal{N}(A)$
\end{enumerate*}

\tinysec{TBD}正交投影

\section{行列式}
\subsection{行列式函数}
\tinysec{定义}定义在全体 $n$ 阶方阵上的函数 $\delta$,如果满足如下性质:
\begin{enumerate*}
    \item 列多线性性:$\delta(\cdots,k\bm{a}_i+k'\bm{a}_i',\cdots)=k\delta(\cdots,\bm{a}_i,\cdots)+k'\delta(\cdots,\bm{a}_i',\cdots)$
    \item 列反对称性:交换任意两列,函数符号反转
    \item 单位化条件:$\delta(I_n)=1$
\end{enumerate*}
则称 $\delta$ 为一个 $n$ 阶\concept{行列式函数},该函数存在且唯一。容易证明 $\det(AB)=\det(A)\det(B)$。\\

\tinysec{定义}\concept{Vandermonde 矩阵}及其行列式
\begin{equation}
    \begin{vmatrix}
        1      & \lambda_1 & \lambda_1^2 & \cdots & \lambda_1^{n-1} \\
        1      & \lambda_2 & \lambda_2^2 & \cdots & \lambda_2^{n-1} \\
        \vdots & \vdots    & \vdots      &        & \vdots          \\
        1      & \lambda_n & \lambda_n^2 & \cdots & \lambda_n^{n-1} \\
    \end{vmatrix}=\prod_{1\le j<i\le n}(\lambda_i-\lambda_j)
\end{equation}

\subsection{行列式的展开式}
\tinysec{定义}给定 $n$ 阶方阵 $A$,令 $A\binom{i}{j}$ 表示从 $A$ 中划去第 $i$ 行第 $j$ 列后得到的 $n-1$ 阶方阵,称 $M_{ij}=\det\left(A\binom{i}{j}\right)$为元素 $a_{ij}$ 的\concept{余子式};而 $C_{ij}=(-1)^{i+j}M_{ij}$ 为元素 $a_{ij}$ 的\concept{代数余子式}。\\

\tinysec{定理}行列式按第一列展开 $\det(A)=a_{11}C_{11}+\cdots+a_{n1}C_{n1}$

\tinysec{定理}令 $A$ 的第 $i$ 列为 $\bm{a}_i$,记第 $j$ 列元素的代数余子式组成的向量 $\bm{c}_j=\begin{bmatrix}
        C_{1j} \\\vdots\\C_{nj}
    \end{bmatrix}$,则
\begin{equation}
    \bm{a}_{j'}^{\mathrm{T}}\bm{c}_j=\begin{cases}
        0,       & j\neq j' \\
        \det(A), & j=j'
    \end{cases}
\end{equation}

\tinysec{定义}对矩阵 $A=[a_{ij}]$,记 $C=[C_{ij}]_{n\times n}$,则 $C^{\mathrm{T}}$ 为 $A$ 的\concept{伴随矩阵},且 $C^{\mathrm{T}}A=\det(A)I_n$。

\section{特征值和特征向量}



\end{document}