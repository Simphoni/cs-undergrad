\documentclass[./main.tex]{subfiles}
\begin{document}
\chapter{线性代数}
\section{线性映射和矩阵}
\subsection{线性映射}
\tinysec{定义}\concept{线性运算}指向量的加法与数乘。\\
\tinysec{定义}\concept{向量空间}为带有线性运算的集合 $\mathbb{R}^m$ 被称为向量空间。\\
\tinysec{定义}映射 $f:\linsp{n}\rightarrow\linsp{m}$ 若满足
\begin{enumerate*}
    \item 任意 $\bm{x},\bm{x}'\in\linsp{n}$,都有 $f(\bm{x}+\bm{x'})=f(\bm{x})+f(\bm{x'})$
    \item 任意 $\bm{x}\in\linsp{n},k\in\linsp{}$, 都有 $f(k\bm{x})=kf(\bm{x})$
\end{enumerate*}
 则 $f$ 为 从 $\linsp{n}$ 到 $\linsp{m}$ 的 \concept{线性映射}。\\
\tinysec{定义}从 $\linsp{n}$ 到 $\linsp{n}$ 的线性映射称为\concept{线性变换}.\\
\tinysec{定义}若线性映射 $f$ 有 $f(\bm{e_i})=\bm{a_i},\bm{e_i}\in\linsp{n}, \bm{a_i}\in\linsp{m}$,则矩阵 $A=[\bm{a_1},\dots,\bm{a_n}]$ 即为标准坐标向量下的 \concept{线性映射的表示矩阵},且满足 $A\bm{e_i}=\bm{a_i}$。\\
\tinysec{定理}(线性映射的线性运算)若矩阵 $\bm{A},\bm{B}$ 表示 $\linsp{n}\rightarrow\linsp{m}$ 的线性映射,则 $\bm{A}+\bm{B}$ 与 $k\bm{A}$ 也是同样范畴上的线性映射。\\
\tinysec{定理}$\bm{AB}$ 表示线性映射 $\bm{A},\bm{B}$ 的复合 $\bm{A}\circ\bm{B}$。需要注意,$\bm{AB}=0$ 也不能推出 $\bm{A}=0$ 或 $\bm{B}=0$。\\
\tinysec{定义}\concept{反对称矩阵}:$A=-A^T$,反对称矩阵对角线必定为0。
\tinysec{定义}若一个上三角矩阵对角线上全为0,则称为\concept{严格上三角矩阵}。\\
\tinysec{定义}将阶梯形矩阵,从下向上消元,并单位化主元,得到的矩阵每个非零行上,主变量为 1 而其他列的元素均为 0,称这个矩阵为 \concept{行简化阶梯形矩阵}。\\
\tinysec{定理}对于 $\bm{v},\bm{w}\in\linsp{m}$,$\bm{v}^T\bm{w} = trace(\bm{w}\bm{v}^T)$。

\subsection{线性方程式组}
\tinysec{定理}对于方程组 $A\bm{x}=\bm{b}$,将 $[A\ \bm{b}]$ 简化为阶梯型后,若
\begin{enumerate*}
    \item $[A\ \bm{b}]$ 阶梯数比 $A$ 多\textbf{1},则方程组无解
    \item $[A\ \bm{b}]$ 阶梯数与 $A$ 相等,则方程组有解
    \begin{enumerate*}
        \item 若阶梯数等于未知数个数,则有唯一解
        \item 若阶梯数小于未知数个数,则有无穷多组解
    \end{enumerate*}
\end{enumerate*}
\tinysec{定义}$A\bm{x}=\bm{b}$ 称为\concept{齐次线性方程组},$\vec{0}$ 为其\concept{平凡解},除此之外的解称为\concept{非平凡解}。

\subsection{可逆矩阵}
\tinysec{定义}设 $A$ 为 $n$ 阶方阵,若存在 $n$ 阶方阵 $B$,使得 $AB=BA=I_n$,则 $A$ 为 \concept{可逆矩阵} 或 \concept{非奇异矩阵},$B$ 为 $A$ 的逆。\\
\tinysec{定理}以下命题等价
\begin{enumerate*}
    \item $A$ 可逆
    \item 任意 $\bm{b}\in\linsp{n}$,$A\bm{x}=\bm{b}$ 的解唯一
    \item 其次方程组 $A\bm{x}=0$ 仅有零解
    \item $A$ 对应的阶梯形矩阵有 $n$ 个主元
    \item $A$ 对应的行简化阶梯形矩阵是 $I_n$
    \item $A$ 能表示为有限个初等矩阵的乘积(即消元至行简化阶梯形矩阵的逆过程)
\end{enumerate*}
\tinysec{定义}若矩阵 $A=[a_{ij}]_{n\times n}$ 对于 $i=1,2,\dots,n$ 都有 $|a_{ii}|>\sum_{j\neq i}|a_{ij}|$,则称其为 \concept{(行)对角占优矩阵}。\\
\tinysec{定理}对角占优矩阵必然可逆。\\
\tinysec{定义}若矩阵 $A$ 通过若干初等\textbf{行变换}可以变为矩阵 $B$,则称 $A,B$ \concept{左相抵}。即存在可逆矩阵 $P$,使得 $PA=B,A=P^{-1}B$。所有和 $A$ 相抵的矩阵中,最简单的是其行简化阶梯形,它被称为 $A$ 的\concept{左相抵标准形}。\\
\tinysec{定理}左相抵构成等价关系。\\
\tinysec{定理}\textbf{(Sherman-Morrison)}设 $A$ 为 $n$ 阶可逆方阵,$\bm{u},\bm{v}$ 为 $n$ 阶向量,则 $A+\bm{uv}^T$ 可逆 $\iff 1+\bm{v}^TA^{-1}\bm{u}\neq 0$,且此时
\begin{equation}
    (A+\bm{uv}^T)^{-1}=A^{-1}-\frac{A^{-1}\bm{uv}^TA^{-1}}{1+\bm{v}^TA^{-1}\bm{u}}
\end{equation}
若将 $\bm{u},\bm{v}$ 改为 $n\times k$ 的矩阵,则类似地有
\begin{equation}
    (A+\bm{uv}^T)^{-1}=A^{-1}-A^{-1}\bm{u}(I_k+\bm{v}^TA^{-1}\bm{u})^{-1}\bm{v}^TA^{-1}
\end{equation}
\subsection{LU分解}
\tinysec{定理}若 $n$ 阶方阵 $A$ 仅通过倍加矩阵做行变化即可化为阶梯形,则存在\concept{单位下三角矩阵}(主对角线均为1)$L$ 与上三角矩阵 $U$,使得 $A=LU$,此即 \concept{LU分解}。\\
\tinysec{定义}方阵 $A$ 左上角的 $k\times k$ 块为第 $k$ 个\concept{顺序主子阵}。\\
\tinysec{定理}可逆矩阵 $A$ 存在 LU 分解,当且仅当 $A$ 的所有顺序主子阵均可逆,此时 LU 分解唯一。(满足在消元过程中不需要行的调换)\\
\tinysec{定理}若可逆矩阵 $A$ 存在 LU 分解,则存在对角线均不为 0 的对角矩阵 $D$、单位下三角矩阵 $L$、单位上三角矩阵 $U$,满足 $A=LDU$,且该分解\textbf{唯一}。此即 \concept{LDU分解}。\\
\tinysec{定理}若可逆对称阵 $A$ 有 LDU 分解,则 $L=U^T$。\\
\tinysec{定理}可逆矩阵 $A$ 存在分解 $A=PLU$,$P$ 为置换矩阵,显然该分解不唯一。
\tinysec{技巧}将 $A$ 分解成 对称阵 $X=\frac{1}{2}(A+A^T)$ 与 反对称阵 $Y=\frac{1}{2}(A-A^T)$。
\section{子空间和维数}
\tinysec{定义}映射 $A:\bm{x}\mapsto A\bm{x}\in\linsp{m\times n}$ 的\concept{像集}
\begin{equation}
    \mathcal{R}(A)=\{A\bm{x}|\bm{x}\in\linsp{n}\}\subseteq\linsp{m}
\end{equation}
显然 $\bm{0}\in\mathcal{R}(A)$总是成立。\\
(\texttt{*}) $\mathcal{R}(A)=\linsp{m}\iff A$ 为满射。\\
\tinysec{定义}映射 $A:\bm{x}\mapsto A\bm{x}\in\linsp{m\times n}$ 的\concept{原像}
\begin{equation}
\mathcal{N}(A)=\{\bm{x}\in\linsp{n}|A\bm{x}=\bm{0}\}\in\linsp{n}
\end{equation}
(\texttt{*}) $\mathcal{N}(A)=\{\bm{0}\}\iff A$ 是单射。
\tinysec{}
\end{document}