\documentclass[./main.tex]{subfiles}
\begin{document}
\chapter{离散数学}
\section{数理逻辑}
\subsection{命题逻辑与运算}
\subsubsection{等值定理}
设 $\Phi(A)$ 是含命题公式 $A$ 的命题公式, $\Phi(B)$是用命题公式 $B$ 置换了 $\Phi(A)$ 中的 $A$ 之后得到的命题公式,如果 A = B,则 $ \Phi(A)=\Phi(B) $.
\subsubsection{常见等值公式}
\begin{align}
	P\rightarrow Q&=\neg P\vee Q & \text{蕴含等值式}\\
    P\rightarrow(Q\rightarrow R)&=(P\wedge Q)\rightarrow R &\text{前提合并提取}\\
    P\leftrightarrow Q&=(P\rightarrow Q)\wedge(Q\rightarrow P) &\text{等价等值式}\\
    P\rightarrow Q&=(\neg Q)\rightarrow(\neg P) &\text{假言易位}\\
    (P\rightarrow Q)\wedge(P\rightarrow \neg Q)&=\neg P &\text{归谬论}\\
    P\rightarrow(Q\rightarrow R)&=Q\rightarrow(P\rightarrow R) &\text{前提交换}
\end{align}
\subsubsection{真值表还原命题公式}
考察真值表中取1的所有 $m$ 行,则可以将命题公式表示为 $$A=Q_1\vee Q_2\vee \dots\vee Q_m = (\ \wedge\ )\vee(\ \wedge\ )\vee\dots\vee(\ \wedge\ )$$其中 $Q_i=(R_1\wedge R_2\wedge\dots\wedge R_n)$,若该行的 $P_i=1$,则 $R_i=P_i$,否则 $R_i=\neg P_i$.
\subsubsection{连结词的完备集}
设 $C$ 是一个联结词的集合,如果可以仅用集合中的连结词构造出所有的 $n$ 元映射,称 $C$ 是完备的联结词集合. 典型的完备集有:$$\{\neg,\wedge\},\ \{\neg,\vee\},\ \{\neg,\rightarrow\},\ \{\uparrow\}\text{(与非)},\ \{\downarrow\}\text{(或非)}$$
\subsubsection{对偶式}
将给定的命题公式 $A$ 中出现的 $\vee$,$\wedge$,$T$,$F$ 分别
以 $\wedge$,$\vee$,$F$,$T$ 代换,得到公式 $A^*$,即为公
式 A的对偶式.\\
\indent 另记 $A^-=A(\neg P_1,\neg P_2,\dots,\neg P_n)$,则有
\begin{gather}
    \neg(A^*)=(\neg A)^*\\
    \neg(A^-)=(\neg A)^-\\
    \neg A=(A^*)^-
\end{gather}(归纳法证明)
\subsubsection{对偶原理}
\begin{itemize}
\item 若 $A=B$,则 $A^*=B^*$.
\begin{equation}
    A=B\implies \neg A=\neg B\implies A^{*-}=B^{*-}\implies A^*=B^*
\end{equation}
\item 若 $A\rightarrow B$ 永真,必有 $B^*\rightarrow A^*$ 永真
\begin{equation}
    A\rightarrow B\implies \neg B\rightarrow\neg A\implies B^{*-}\rightarrow A^{*-}\implies B^*\rightarrow A^*
\end{equation}
\item $A$ 永真 $\Leftrightarrow A^-$ 永真($\neg A$ 永真 $\Leftrightarrow A^*$ 永真);$A$ 永真 $\Leftrightarrow$ $A^*$ 矛盾
\end{itemize}
\subsubsection{范式存在定理}
命题变项及其否定式统称\textbf{文字},由文字的合(析)取所组成的公式称为\textbf{合(析)取式}\\
\indent \ding{72}形如 $A_1\vee A_2\vee\dots\vee A_n$ 的公式为\textbf{析取范式},其中 $A_i$ 为合取式. \\
\indent \ding{72}形如 $A_1\wedge A_2\wedge\dots\wedge A_n$ 的公式为\textbf{合取范式},其中 $A_i$ 为析取式. \\
\indent \textbf{(范式的关键在于嵌套只有两层)}\\
\indent \textbf{任一}命题公式\textbf{都存在}与之等值的合取范式和析取范式,但命题公式的合取范式和析取范式并不唯一.\hfill(\textsc{proof})
\subsubsection{主范式}
\textbf{极小项}:指合取式 $Q_1\wedge Q_2\wedge\dots Q_n$,$Q_i=P_i\ \text{或}\ \neg P_i$\\
\indent \textbf{主析取范式}:每个\underline{一级命题变项}都是极小项,且两两不相同\\
\indent \textbf{极大项}:指析取式 $Q_1\vee Q_2\vee\dots Q_n$,$Q_i=P_i\ \text{或}\ \neg P_i$\\
\indent \textbf{主合取范式}:每个\underline{一级命题变项}都是极大项,且两两不相同
\subsubsection{极小项性质}
\begin{enumerate}
    \item 任意极小项仅在\textbf{一个}解释下为真
    \item 极小项两两不等值,两两不同时为真
    \item 对于任意一个解释,$2^n$ 个极小项中\textbf{有且仅有}一个为真
    \item 若一个基本变项不在于合取式中,则可以拆分成 $(Q_1\wedge Q_2\wedge P_3)\vee(Q_1\wedge Q_2\wedge \neg P_3)$
\end{enumerate}
\subsubsection{主范式定理}
\indent \textbf{任一}命题公式\textbf{都存在}与之等值的\textbf{唯一}主合取范式和主析取范式.\hfill(与多项式插值类似)
\subsubsection{两种主范式的转换}
引入二进制表示,如 $m_5=P_1\wedge\neg P_2\wedge P_3$(极小项),$M_5=P_1\vee\neg P_2\vee P_3$(极大项)\\
\indent 设有基于 $n$ 个命题变量的主合取范式 $A=\bigwedge M_{i_1,\dots,i_k}$,则由于 $M_i\ (0\le i<2^n)$ 中有且仅有一个为假,故
\begin{align}
    \neg A&=\bigwedge M_{(0,1,\dots,2^n-1)\backslash(i_1,\dots,i_k)}\notag \\
    A&=\left(\bigwedge M_{(0,1,\dots,2^n-1)\backslash(i_1,\dots,i_k)}\right)^{*-}\notag
    \\&=\bigvee m_{\text{bitinv}[(0,1,\dots,2^n-1)\backslash(i_1,\dots,i_k)]}
\end{align}
\end{document}