\documentclass[./main.tex]{subfiles}
\begin{document}
\chapter{概率论}
\section{事件的概率}
\subsection{概率的公理化定义}
\noindent 设 $\Omega$ 为样本空间,定义事件集类 $\mathscr{J}\subset2^\Omega$。定义 $P:\mathscr{J}\rightarrow\mathbb{R}$ 满足三条公理
\begin{enumerate*}
    \item $P(A)\ge 0,\forall A\in\mathscr{J}$
    \item $P(\Omega)=1$
    \item $P(\sum_{i=1}^{\infty}A_i)=\sum_{i=1}^{\infty}P(A_i),\forall A_i\in\mathscr{J},A_iA_j=\phi,\forall i\neq j$
\end{enumerate*}
则称 $P$ 为\underline{\textbf{概率函数}},$(\omega,\mathscr{J},P)$ 为\underline{\textbf{概率空间}}
\subsubsection{错位排序}
$A_i$ 表示第 $i$ 个数恰好在原位上,则每个数\textbf{都不在}原位的概率为
\begin{align}
    P&=1-P(A_1+A_2+\dots+A_n)\notag\\
    &=1-\sum_{i_1\le\dots\le i_r}(-1)^{r+1}P(A_{i_1}\dots A_{i_r})\notag\\
    &=1-\sum_{r=1}^n(-1)^{r+1}\binom{n}{r}\frac{(n-r)!}{n!}(=\frac{1}{r!})\notag\\
    &=1-[1-\frac{1}{2!}+\frac{1}{3!}+\dots+(-1)^{n+1}\frac{1}{(n)!}](\rightarrow \frac{1}{e})
\end{align}
\subsection{条件概率}
设 $A,B\in\mathscr{J}$,定义 \underline{\textbf{条件概率}}为
\begin{equation}
    P(A|B)\triangleq\frac{P(AB)}{P(B)}\ \ (P(B)>0)
\end{equation}
容易证明,$$\widetilde{P}(A)=P(A|B):\mathscr{J}\rightarrow\mathbb{R}$$ 也是概率函数。
\subsection{独立事件}
\subsubsection{两个事件的独立}
若 $P(AB)=P(A)P(B)$ 则称 $A,B$ 相互独立。此时有 $P(A)=P(A|B)$,并且能\textbf{推出 $A$ 与 $B^C$ 独立}。
\subsubsection{多个事件的独立}
若对 $A_1,A_2,\dots$ 为可数个事件,若从中\textbf{任取有限个事件} $A_{i_1},\dots,A_{i_m}$ 都有 $$P(A_{i_1}\cdots A_{i_m})=P(A_{i_1})\cdots P(A_{i_m})$$,则称 $A_1,A_2,\dots$ 相互独立。
\subsubsection{条件独立}
若 $P(AB|E)=P(A|E)P(B|E)$,则称事件 $A,B$ 关于事件 $E$ 条件独立。\\
\indent \underline{\textbf{注意:条件独立不能推出独立,独立也不能推出条件独立}}
\subsection{贝叶斯(Bayes)公式}
定义 $\Omega$ 的一个分割 $\{B\}$ 满足 $\sum_i B_i=\Omega$ 且 $B_iB_j=\phi,\forall i\neq j$。则
\begin{gather}
    P(A)=P(A\sum_iB_i)=\sum_iP(B_i)P(A|B_i)\\
    P(B_j|A)=\frac{P(AB_j)}{P(A)}=\frac{P(B_j)P(A|B_j)}{\sum_iP(B_i)P(A|B_i)}\label{equ:bayes}
\end{gather}
\eqref{equ:bayes} 即为Bayes公式。
\section{随机变量}
\subsection{1维随机变量}
定义实值函数 $X(\omega):\Omega\rightarrow\mathbb{R}$,该映射给样本空间中的每个试验结果一个对应的数值,\underline{\textbf{随机变量}}即定义为试验结果的一个实值函数,也可以通过\textbf{随机变量的函数}定义另一个随机变量。{\kaishu(默认用大写字母表示随机变量,小写字母表示实数。)}
\subsection{随机变量的概率}
$\forall I\in\mathbb{R}$ 且 $I$ 为可测集,利用数值所对应的事件定义概率函数 $P_X:\mathbb{R}\rightarrow\mathbb{R}$
$$P_X(X\in I)\triangleq P(X^{-1}(I))$$
\subsection{(累积)分布函数(cdf)}
\noindent 定义为 $F(x)\triangleq P(X\le x), x\in\mathbb{R}$,则 $P(a<X\le b)=F(b)-F(a)$。cdf有以下性质:
\begin{enumerate*}
    \item $F(x)$ 单调(非严格)递增
    \item $\lim\limits_{x\rightarrow-\infty}F(x)=0,\lim\limits_{x\rightarrow+\infty}F(x)=1$
    \item $F(x)$ 右连续
\end{enumerate*}
以上性质也是 $F(x)$ 成为cdf的充要条件。
\subsection{离散分布}
\subsubsection{概率质量函数(pmf)}
定义为 $f(x)=P(X=x), \forall x\in\mathbb{R}$。
\subsubsection{期望和方差}
期望 $E(X)\triangleq \sum_ix_if(x_i)$;方差 $Var(X)\triangleq\sum_i(x_i-E(X))^2f(x_i)=E(X-E(X)^2)$。
\begin{enumerate*}
    \item 若用 $X$ 的函数 $g(X)$ 定义 $Y$,则 $E(g(X))=\sum_{i}g(x_i)f(x_i)\neq g(E(X))$
    \item $E(X+Y)=E(X)+E(Y)$;若 $X,Y$ 独立,则 $Var(X+Y)=Var(X)+Var(Y)$
\end{enumerate*}
\subsection{常见离散分布}
\subsubsection{Bernoulli分布}
\begin{itemize*}
    \item $X=\begin{cases}1,&p\\0,&1-p\end{cases}$,记为$X\sim B(p)$
    \item $E(X)=p$,$Var(X)=p(1-p)$
\end{itemize*}
\subsubsection{二项分布}
\begin{itemize*}
    \item $P(X=k)=\binom{n}{k}p^k(1-p)^{n-k}$,记为 $X\sim B(n,p)$
    \item $X$ 为 $n$ 次伯努利试验中试验成功的次数
    \item $E(X)=np$,$Var(X)=np(1-p)$
\end{itemize*}
\subsubsection{Poisson分布}
\begin{itemize*}
    \item $f(X=k)=e^{-\lambda}\frac{\lambda^k}{k!}$,记为 $X\sim P(\lambda)$
    \item $E(X)=\lambda$,$Var(X)=\lambda$
    \item 对于 $X\sim B(n,p)$,$n$ 非常大而 $np$ 较小时,$X$ 近似服从 $P(np)$
    \item $P(\lambda)$ 多用于当 $X$ 表示一定时间内出现的小概率时间的次数时
\end{itemize*}
\subsection{连续分布}
\subsubsection{概率密度函数(pdf)}
若存在 $f\ge 0$,使得 $\forall$ 可测集 $I\subset R$ 都有
$$P(X\in I)=\int_If(x)\d{x}$$,则 $X$ 为连续型随机变量,$f$ 为 $X$ 的概率密度函数
\begin{enumerate*}
    \item $P(X=a)\equiv 0, \forall a\in\mathbb{R}$
    \item $E(X)=\int_{-\infty}^{\infty}xf(x)\d{x}$
    \item $Var(X)=\int_{-\infty}^{\infty}(x-E(X))^2f(x)\d{x}=E(X^2)-E(X)^2$
\end{enumerate*}
\subsection{常见连续分布}
\subsubsection{均匀分布}
\begin{itemize*}
    \item $f(x)=\begin{cases}\frac{1}{b-a},&x\in(a,b)\\0,&otherwise\end{cases}$,记为$X\sim U(a,b)$
    \item $E(X)=\frac{a+b}{2}$,$Var(X)=\frac{(b-a)^2}{12}$
\end{itemize*}
\subsubsection{正态分布}
\begin{itemize*}
    \item $f(x)=\frac{1}{\sqrt{2\pi}\sigma}e^{-\frac{(x-\mu)^2}{2\sigma^2}},\forall x\in\mathbb{R}$,记为 $X\sim N(\mu,\sigma^2)$
    \item $E(X)=\mu,Var(X)=\sigma^2$
    \item $X\sim N(\mu,\sigma^2)\implies Y=\frac{X-\mu}{\sigma}\sim N(0,1)$
\end{itemize*}
\subsubsection{指数分布}
\begin{itemize*}
    \item $f=\begin{cases}\lambda e^{-\lambda x},&x>0\\0,&x\le 0\end{cases}$
    \item $E(X)=\frac{1}{\lambda},Var(X)=\frac{1}{\lambda^2}$
    \item $F(X)=1-e^{-\lambda x}$
    \item $X$ 一般刻画寿命或者等待时间
    \item 无记忆性:$$P(X>x+t|X>x)=\frac{P(X>x+t)}{P(X>x)}=\frac{e^{-\lambda(x+t)}}{e^{-\lambda x}}=e^{-\lambda t}=f(t)$$
\end{itemize*}
\section{联合分布}
\subsection{随机向量}
$(X_1,X_2,\dots,X_n):\Omega\rightarrow\mathbb{R}^n$,其中 $X_i$ 为随机变量,定义(联合)累积分布函数
$$F(x_1,\dots,x_n)\triangleq P(X_1\le x_1,\dots,X_n\le x_n),(x_1,\dots,x_n)\in\mathbb{R}^n$$
\subsection{离散分布}
若 $\forall 1\le i\le n,X_i$ 为离散型随机变量,则 $(X_1,\dots,X_n)$ 为离散型随机向量。定义pmf为 $f(x_1,\dots,x_n)\triangleq P(X_1=x_1,\dots,X_n=x_n)$。
\subsection{连续分布}
若对一切可测集 $Q\subset\mathbb{R}^n$,$P((X_1,\dots,X_n)\in Q)=\int_Qf(x_1,\dots,x_n)\d{x_1}\cdots\d{x_n}$,则 $(X_1,\dots,X_n)$ 为连续型随机向量,$f$ 为其pdf。\\
\indent 以 $n=2$ 为例,$F(a,b)=\int_{-\infty}^a\int_{-\infty}^bf(x,y)\d{x}\d{y}$,反过来有 $f(a,b)=\pds{^2F}{x\p{y}}$
\subsubsection{二元正态分布}
$X,Y\sim N(\mu_1,\mu_2,\sigma_1^2,\sigma_2^2,\rho),|\rho|<1$
\begin{equation}
    f(x,y)=\frac{1}{2\pi\sigma_1\sigma_2}\frac{1}{\sqrt{1-\rho^2}}\exp\left\{-\frac{1}{2(1-\rho^2)}\left[\left(\frac{x-\mu_1}{\sigma_1}\right)^2+\left(\frac{y-\mu_2}{\sigma_2}\right)^2-2\rho\frac{x-\mu_1}{\sigma_1}\frac{y-\mu_2}{\sigma_2}\right]\right\}
\end{equation}
\subsection{边际分布}
对于的连续随机变量,定义 $X_i$ 的边际累积分布函数(cdf)
\begin{equation}
    F_i(x)\triangleq P(X_i\le x)=P(X_i\le x,-\infty<X_j<\infty(j\neq i))
\end{equation}
以 $n=2$ 为例,$F_X(x)=\lim\limits_{y\rightarrow\infty}P(X\le x,Y\le y)=\lim\limits_{y\rightarrow\infty}F(x,y)$。\\
定义$X_i$ 的边际密度函数(pdf)
 $$f_X(x)=\int_{-\infty}^{\infty}f(x,y)\d{y}$$
\subsection{条件分布}
对于 $n=2$ 的连续型随机变量,求其条件密度函数
\begin{align}
    P(X\le x|y\le Y\le y+\d{y})&=\frac{\int_{-\infty}^x\left(\int_y^{y+\d{y}}f(t,s)\d{q}\right)\d{p}}{\int_{y}^{y+\d{y}}f_Y(q)\d{q}}\notag\\
    P(X=x|y\le Y\le y+\d{y})&=\frac{\int_y^{y+\d{y}}f(t,s)\d{q}}{\int_{y}^{y+\d{y}}f_Y(q)\d{q}}\notag\rightarrow\frac{f(x,y)}{f_Y(y)}\notag(\d{y}\rightarrow 0)\\
    f_X(x|y)&\triangleq\frac{f(x,y)}{f_Y(y)}
\end{align}
定义 $f_X(x|y)$ 的累积分布函数(cdf)为 $F(a|y)=\int_{-\infty}^a f_X(x|y)\d{x}$
\begin{enumerate}[(1)]
    \item (乘法法则)$f(x,y)=f_X(x|y)f_Y(y)=f_Y(y|x)f_X(x)$
    \item (全概率公式)$f_X(x)=\int_{-\infty}^{+\infty}f(x,y)\d{y}=\int_{-\infty}^{+\infty}f_X(x|y)f_Y(y)\d{y}$
\end{enumerate}
对于二元正态分布,
\begin{equation}
    f_Y(y|x)=\frac{1}{\sqrt{2\pi}\sigma_2}\frac{1}{2\sqrt{1-\rho^2}}\exp\left\{-\frac{[y-(\mu_2+\rho\frac{\sigma_2}{\sigma_1}(x-\mu_1))]^2}{2(1-\rho^2)\sigma_2^2}\right\}
\end{equation}
即 $\bm{Z=(Y|x)\sim N(\mu_2+\rho\frac{\sigma_2}{\sigma_1}(x-\mu_1),(1-\rho^2)\sigma_2^2)}$
\subsection{独立性}
若 $F(x_1,\dots,x_n)=F_1(x_1)\cdots F_n(x_n),\forall x_1,\dots,x_n\in\mathbb{R}$,则称 $X_1,\dots,X_n$ 相互独立。该条件完全等价于 $f(x_1,\dots,x_n)=f_1(x_1)\cdots f_n(x_n)$。
\begin{enumerate*}
    \item 若 $X_1,\dots,X_n$ 相互独立,则 $Y_1=g_1(X_1,\dots,X_m)$ 与 $Y_2=g_2(X_{m+1},\dots,X_n)$ 相互独立
    \item 若 pdf $f(x_1,\dots,x_n)=g_1(x_1)\cdots g_n(x_n),\forall x_1,\dots,x_n\in\mathbb{R}$,则 $X_1,X_2,\dots,X_n$ 相互独立,且 $f_i$ 与 $g_i$ 相差常数倍 
\end{enumerate*}
\subsection{多个随机变量的函数}
\subsubsection{\ding{72}密度函数变换法}
设存在由随机变量 $X1,X2$ 到 $Y_1,Y_2$ 的可逆映射 $\begin{cases}Y_1=g_1(X_1,X_2)\\Y_2=g_2(X_1,X_2)\end{cases}$,逆为 $\begin{cases}X_1=h(Y_1,Y_2)\\X_2=h(Y_1,Y_2)\end{cases}$。对于 $X_1,X_2$ 上的区域 $A$,若其在 $Y_1,Y_2$ 上的映射为 $B$,则显然有 $$P((X_1,X_2)\in A)=P((Y_1,Y_2)\in B)$$
\begin{align*}
    P((Y_1,Y_2)\in B)&=\int_B\mathscr{F}(y_1,y_2)\d{y_1}\d{y_2}\\
    P((X_1,X_2)\in A)&=\int_Af(x_1,x_2)\d{x_1}\d{x_2}=\int_Bf(h(y_1),h(y_2))\bigg|\frac{D{(x_1,x_2)}}{D{(y_1,y_2)}}\bigg|\d{y_1}\d{y_2}
\end{align*}
因此
\begin{equation}
    \mathscr{F}(Y_1,Y_2)=f(h(Y_1),h(Y_2))\bigg|\frac{D{(x_1,x_2)}}{D{(y_1,y_2)}}\bigg|
\end{equation}
\section{随机变量的数学特征}
\subsection{期望}
\begin{enumerate*}
    \item 刻画分布的集中趋势
    \item $X_1,\dots,X_n$ 独立时,$E(X_1X_2\cdots X_n)=E(X_1)\cdots E(X_n)$
\end{enumerate*}
\subsection{分位数}
$\forall \alpha\in(0,1)$,若 $P(X\le a)\ge\alpha$ 且 $P(X\ge a)\ge 1-\alpha$,则称 $X=a$ 为 $X$ 的 \textbf{$\text{下}\alpha\text{-分位数}$}。
\begin{enumerate*}
    \item 若 cdf 连续,则 $F(a)=\alpha$
    \item 分位数不一定唯一
\end{enumerate*}
\subsection{方差}
\begin{enumerate*}
    \item 刻画分布的集中程度
    \item $Var(cX)=c^2Var(X)$
    \item $Var(X+Y)=Var(X)+Var(Y)+2E[(X-\mu_1)(Y-\mu_2)]$
\end{enumerate*}
\subsection{协方差与相关系数}
\noindent 定义协方差 $Cov(X,Y)\triangleq E[(X-\mu_1)(Y-\mu_2)]$
\begin{enumerate*}
    \item $Cov(X,X)=Var(X)$
    \item $Cov(X,Y)=E(XY)-E(X)E(Y)$
    \item $Cov(aX_1+bX_2+c,Y)=aCov(X_1,Y)+bCov(X_2,Y)$
\end{enumerate*}
定义相关系数 $Corr(X,Y)\triangleq \frac{Cov(X,Y)}{\sigma_1\sigma_2}=E(\frac{X-\mu_1}{\sigma_1}\frac{Y-\mu_2}{\sigma_2})=\rho$
\begin{enumerate*}
    \item $X,Y$ 独立 $\implies Cov(X,Y)=0$,反之不一定
    \item $|Corr(X,Y)|\le 1$
    \item $\rho$ 实际上是\textbf{线性}相关系数,不能表达高维的相关关系
\end{enumerate*}
\subsection{矩(Moment)}
\tinysec{定义}将 $E[(X-C)^n]$ 称为 $X$ 关于 $C$ 的 $n$ 阶矩。当 $C=E(X)$ 时,称为中心矩;当 $C=0$ 时,称为原点矩;标准化后的矩 $E[\left(\frac{x-\mu}{\sigma}\right)^n]$ 称为 $n$ 阶标注矩。
\subsubsection{偏度系数}
\tinysec{定义}3阶标准矩又称偏度系数。
\begin{enumerate*}
    \item 偏度系数小于零,左偏;大于零,右偏
    \item 相对于5阶以上的奇数阶矩,3阶矩容易计算且噪声的影响小
\end{enumerate*}
\subsubsection{峰度系数}
\tinysec{定义}4阶标准矩又称峰度系数。
\begin{enumerate*}
    \item 正态分布峰度恒为3
    \item 超值峰度定义为 $E[\left(\frac{X-\mu}{\sigma}\right)^4]$
    \item 超值峰度为正,一般相对于正态分布峰更尖,尾部更扁
\end{enumerate*}
\subsection{矩母函数}
\tinysec{定义}(moment generating function, mgf) $M_X(t)\triangleq E(e^{tX})$ 若在 $t=0$ 的某个邻域内 $M_X(t)$ 存在,则称其为 $X$ 的矩母函数,否则称 $X$ 的矩母函数不存在。\textbf{注意标明 $\bm{t}$ 的取值范围。}\\
对于 $X\sim Exp(\lambda)$,
\begin{equation}
    M_X(t)=\int_0^{\infty}e^{tx}\lambda e^{-\lambda x}\d{x}=\frac{\lambda}{\lambda -t},t<\lambda
\end{equation}
对于 $X\sim N(\mu,\sigma^2)$,
\begin{equation}
    M_X(t)=e^{\frac{\sigma^2t^2}{2}+\mu t}
\end{equation}
对于 $Y=aX+b$,
\begin{equation}
    M_Y(t)=E(e^{aX+b})=e^{tb}E(e^{taX})=e^{tb}M_X(ta)
\end{equation}
\tinysec{定理}矩母函数确定矩:\begin{equation}E(X^n)=M_X^{(n)}(0)\end{equation}
\begin{proof}
    \begin{align*}
        M_X(t)&=\sum_{n\ge 0}M_X^{(n)}(0)\frac{t^n}{n!}\\
        M_X(t)&=E(e^{tx})=E\left(\sum_{n\ge 0}\frac{(tx)^n}{n!}\right)=\sum_{n\ge 0}E(t^n)\frac{x^n}{n!}
    \end{align*}
    由于 $M_X(t)$ 泰勒展开唯一,故 $M_X^{(n)}=E(t^n)$
\end{proof}
\tinysec{例}对于$X\sim N(0,1)$,
$$M_X(t)=e^{\frac{t^2}{2}}=\sum_{n\ge 0}\frac{t^{2n}}{2^nn!}=\sum_{n\ge 0}\frac{(2n)!}{2^nn!}\frac{t^{2n}}{(2n)!}$$
故 $E(X^{2n})=\frac{(2n!)}{2^nn!},E(X^{2n-1})\equiv 0$\\
\tinysec{定理}矩母函数确定分布。若存在 $a>0$,使得 $M_X(t)=M_Y(t),t\in(-a,a)$,则 $X,Y$ 同分布。\\
\tinysec{REMARK}矩存在的时候,矩母函数不一定存在(即泰勒展式不收敛),因此各阶矩完全相同也无法说明两个随机变量同分布。\\
\tinysec{例}取服从对数正态分布的变量 $Y$ 于另一变量 $Z$
\begin{align*}
    y&=f_1(x)=\frac{1}{\sqrt{2\pi x}}e^{-\frac{\log^2x}{2}},x>0\\
    z&=f_2(x)=f_1(x)[1+\sin(2\pi\log x)],x>0
\end{align*}
则它们的 $n$ 阶矩存在以下关系:
$$E(Z^n)-E(Y^n)=\int_0^{\infty}x^nf_1(x)\sin(2\pi\log x)\d{x}=T$$
令 $t=\log x-n$,则 $x=e^{t+n}$
\begin{align*}
    T&=\frac{1}{\sqrt{2\pi}}\int_{-\infty}^{\infty}e^{n(t+n)}e^{-\frac{1}{2}(t+n)}e^{-\frac{1}{2}(t+n)^2}\sin(2\pi(t+n))e^{t+n}\d{t}\\
    &=\frac{1}{\sqrt{2\pi}}\int_{-\infty}^{\infty}\exp\left[\frac{1}{2}(n+t)(n+t+1)\right]\d{t}\\
    &=\frac{e^{-\frac{1}{8}}}{\sqrt{2\pi}}\int_{-\infty}^{\infty}\exp\left[\frac{1}{2}(n+t+\frac{1}{2})^2\right]\d{t}=0
\end{align*}
因此对一切 $n$,$Y,Z$ 的 $n$ 阶矩相等。但是由于 $$e^{tx}f_1(x)=\frac{1}{\sqrt{2\pi}}e^{tx-\frac{\log^2x}{2}}$$ 对于一切的 $t>0$ 都有 $\lim\limits_{x\rightarrow\infty}e^{tx}f_1(x)=+\infty$,因此无法积分,$M_Y(t)$不存在。故即使各阶矩完全相同,实际上也不是同分布。
\subsubsection{独立随机变量和的分布}
对于相互独立的 $X_1,X_2,\dots,X_n$,有
\begin{equation}
    M_{X_1+\cdots+X_n}(t)=M_{X_1}(t)\cdots M_{X_n}(t)
\end{equation}
\indent 对于联合分布 $(X_1,X_2,\dots,X_n)$,定义矩母函数为多元函数
\begin{equation}
    M_{X_1,\dots,X_n}(t_1,t_2,\dots,t_n)=E(e^{t_1X_1+\cdots+t_nX_n})
\end{equation}
若该函数在原点的邻域 $B(0,\sigma),\sigma>0$ 内有定义,则称联合分布的矩母函数存在;若两组联合分布的矩母函数在 $B(0,\sigma)$ 内恒等,则这两组联合分布同分布。
\subsection{条件期望}
\tinysec{定义}$$E(Y|X=x)=\begin{cases}\sum_iy_iP(Y=y_i|X=x)\\\int_{-\infty}^{\infty}f_Y(y|X=x)\d{y}\end{cases}$$
\tinysec{定理}$E(Y)=E(E(Y|X))$\\
\tinysec{定理}$E((Y-g(X))^2)\ge E[(Y-E(Y|X))^2]$,这意味着 $E(Y|X)$ 是均方误差意义下的最优预测。均方最优:$E[(Y-c)^2]\ge E[(Y-E(Y))^2]$,这是因为 $\pds{E[(Y-c)^2]}{c}=2E[Y-c]\big|_{c=E(Y)}=0$。
\section{不等式与极限定理}
\subsection{概率不等式}
\subsubsection{Markov不等式}
若 $Y\ge 0$,则 $\forall a>0$,有
\begin{equation}\label{neq:markov}
    P(Y\ge a)\le \frac{E(Y)}{a}
\end{equation}
\begin{proof}令 $I=\begin{cases}
    1, Y\ge a\\0,Y<a
\end{cases}$,则不论 $Y$ 的取值,均有 $I\le \frac{Y}{a}$\\故 $P(Y\ge a)=E(I)\le E(\frac{Y}{a})=\frac{E(Y)}{a}$
\end{proof}
\subsubsection{Chebyshef不等式}
若 $Var(Y)$ 存在,则 $\forall a>0$
\begin{equation}
    E(|Y-E(Y)|\ge a)\le \frac{Var(Y)}{a^2}
\end{equation}
\begin{proof}
    在\eqref{neq:markov}中代入 $P[|Y-E(Y)|\ge a]=P[(Y-E(Y))^2\ge a^2]=E[Var(Y)\ge a^2]$ 即可。
\end{proof}
\subsubsection{Chernoff不等式}
$\forall a>0,t>0$ 有 
\begin{equation}
    P(Y\ge a)\le \frac{E(e^{tY})}{e^{ta}}
\end{equation}
\subsection{大数定律}
设 $X_1,X_2,\dots,X_n$ 独立同分布,$\bar{X}=\frac{1}{n}\sum_{k=1}^nX_k$ ,则\\
\subsubsection{Khinchin弱大数定律}$\forall \varepsilon>0$ 有
\begin{equation}
    \lim\limits_{n\rightarrow\infty}P(|\bar{X}-\mu|<\varepsilon)=1
\end{equation}
\begin{proof}
    $$P(|\bar{X}-\mu|\ge\varepsilon)\le\frac{Var(\bar{X})}{\varepsilon^2}=\frac{\sigma^2}{n\varepsilon^2}\rightarrow 0(n\rightarrow\infty)$$
\end{proof}
若 $P(|\bar{X}-\mu|\ge\varepsilon)\le\alpha$,则称 $\alpha$ 为置信水平,$\varepsilon$ 为精度。
\subsubsection{Kolmogorov强大数定律}
$\bar{X}$ 几乎必然收敛到 $\mu$
\begin{equation}
    \forall\varepsilon>0,\ P(\lim\limits_{n\rightarrow\infty}|\bar{X}-\mu|<\varepsilon)=1
\end{equation}
\subsection{中心极限定理(CLT)}
若 $E(X_i),Var(X_i)$ 均存在,则
\begin{equation}
    \lim\limits_{n\rightarrow\infty}P(\frac{X_1+\cdots+X_n-n\mu}{\sqrt{n}\sigma}\le x)=\Phi(X)\sim N(0,1),\forall x\in\mathbb{R}
\end{equation}
\end{document}